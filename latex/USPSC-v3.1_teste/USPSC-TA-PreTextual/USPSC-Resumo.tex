%% USPSC-Resumo.tex
\setlength{\absparsep}{18pt} % ajusta o espaçamento dos parágrafos do resumo		
\begin{resumo}
	\begin{flushleft} 
			\setlength{\absparsep}{0pt} % ajusta o espaçamento da referência	
			\SingleSpacing 
			\imprimirautorabr~~\textbf{\imprimirtituloresumo}.	\imprimirdata. \pageref{LastPage}p. 
			%Substitua p. por f. quando utilizar oneside em \documentclass
			%\pageref{LastPage}f.
			\imprimirtipotrabalho~-~\imprimirinstituicao, \imprimirlocal, \imprimirdata. 
 	\end{flushleft}
\OnehalfSpacing 			
 O resumo deve ressaltar o  objetivo, o método, os resultados e as conclusões do documento. A ordem e a extensão  destes itens dependem do tipo de resumo (informativo ou indicativo) e do  tratamento que cada item recebe no documento original. O resumo deve ser
 precedido da referência do documento, com exceção do resumo inserido no
 próprio documento. (\ldots)  Salientamos que algumas Unidades exigem o titulo dos trabalhos acadêmicos em inglês, tornando necessário a inclusão das referências nos resumos e abstracts, o que foi adotado no \textbf{Modelo para TCC em \LaTeX\ utilizando a classe USPSC} e no \textbf{Modelo para teses e dissertações em \LaTeX\ utilizando a classe USPSC}. As palavras-chave devem figurar logo abaixo do  resumo, antecedidas da expressão Palavras-chave:, separadas entre si por  ponto e finalizadas também por ponto \cite{nbr6028}.
 

 \textbf{Palavras-chave}: LaTeX. Classe USPSC. Tese. Dissertação. Trabalho de conclusão de curso (TCC). 
\end{resumo}