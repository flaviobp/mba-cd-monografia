%% USPSC-TCC-pre-textual-ICMC.tex
%% Camandos para defini��o do tipo de documento (tese ou disserta��o), �rea de concentra��o, op��o, pre�mbulo, titula��o 
%% referentes ao Programa de P�s-Gradua��o o IFSC
\instituicao{Instituto de Ci\^encias Matem\'aticas e de Computa\c{c}\~ao, Universidade de S\~ao Paulo}
\unidade{INSTITUTO DE CI\^ENCIAS MATEM\'ATICAS E DE COMPUTA\c{C}\~AO}
\unidademin{Instituto de Ci\^encias Matem\'aticas e de Computa\c{c}\~ao}
\universidademin{Universidade de S\~ao Paulo}
\notafolharosto{Vers\~ao original}
%Para vers�o original em ingl�s , comente do comando/declara��o acima (inclua % antes do comando acima) 
%                       e tire a % do comando/declara��o abaixo no idioma do texto
%\notafolharosto{Original version} 
%Para vers�o revisada, comente do comando/declara��o acima (inclua % antes do comando acima) 
%                       e tire a % do comando/declara��o de um dos comandos abaixo em conformidade com o idioma do texto
%\notafolharosto{Vers\~ao revisada}
%\notafolharosto{Final version}

% ---
% dados complementares para CAPA e FOLHA DE ROSTO
% ---
\universidade{UNIVERSIDADE DE S\~AO PAULO}
\titulo{Aplica\c{c}\~ao de Modelos de Aprendizagem de M\'aquina para a Previs\~ao da Infla\c{c}\~ao no Brasil} 
\titleabstract{Model for TCC in \LaTeX\ using the USPSC class to the ICMC}
% para a vers�o em ingl�s, utilize os comandos abaixo
%idioma{eng}
%\titulo{Model for theses and dissertations in LaTeX using the USPSC class to the ICMC}
%\titleabstract{Modelo para teses e disserta\c{c}\~oes em LaTeX utilizando a classe USPSC para o ICMC}

\autor{Fl\'avio Bezerra Pereira}
\autorficha{Pereira, Fl\~avio Bezerra}
\autorabr{PEREIRA, F. B.}

\cutter{S856m}
% Para gerar a ficha catalogr�fica sem o C�digo Cutter, basta 
% incluir uma % na linha acima e tirar a % da linha abaixo
%\cutter{ }

\local{S\~ao Carlos}
\data{2024}
% Quando for Orientador, basta incluir uma % antes do comando abaixo
%\renewcommand{\orientadorname}{Orientadora:}
% Quando for Coorientadora, basta tirar a % utilizar o comando abaixo
%\newcommand{\coorientadorname}{Coorientador:}
\orientador{Prof. Dr. Vicente Garibay Cancho}
\orientadorcorpoficha{orientador Vicente Garibay Cancho}
\orientadorficha{Garibay Cancho, Vicente, orient}
%Se houver co-orientador, inclua % antes das duas linhas (antes dos comandos \orientadorcorpoficha e \orientadorficha) 
%          e tire a % antes dos 3 comandos abaixo
%\coorientador{Prof. Dr. Jo\~ao Alves Serqueira}
%\orientadorcorpoficha{orientadora Elisa Gon\c{c}alves Rodrigues ;  co-orientador Jo\~ao Alves Serqueira}
%\orientadorficha{Rodrigues, Elisa Gon\c{c}alves, orient. II. Serqueira, Jo\~ao Alves, co-orient}

\notaautorizacao{AUTORIZO A REPRODU\c{C}\~AO E DIVULGA\c{C}\~AO TOTAL OU PARCIAL DESTE TRABALHO, POR QUALQUER MEIO CONVENCIONAL OU ELETR\^ONICO PARA FINS DE ESTUDO E PESQUISA, DESDE QUE CITADA A FONTE.}
\notabib{Ficha catalogr\'afica elaborada pela Biblioteca Prof. Achille Bassi, ICMC/USP, com os dados fornecidos pelo(a) autor(a)}

\newcommand{\programa}[1]{

% MBACD ==========================================================================
    \ifthenelse{\equal{#1}{MBACD}}{
     		\tipotrabalho{Monografia (MBA em Ci\^encias de Dados)}
        \area{Ci\^encias de Dados}
				%\opcao{Nome da Op��o}
        % O preambulo deve conter o tipo do trabalho, o objetivo, 
				% o nome da institui��o, a �rea de concentra��o e op��o quando houver
		\preambulo{Trabalho de conclus\~ao de curso apresentado ao Centro de Ci\^encias Matem\'aticas Aplicadas \`a Ind\'ustria do Instituto de Ci\^encias Matem\'aticas e de Computa\c{c}\~ao, Universidade de S\~ao Paulo, como parte dos requisitos para conclus\~ao do MBA em Ci\^encias de Dados.}
		\notaficha{Monografia (MBA em Ci\^encias de Dados)}
	}{
% Outros
		\tipotrabalho{Disserta\c{c}\~ao/Tese (Mestrado/Doutorado)}
		\area{Nome da \'Area}
		\opcao{Nome da Op\c{c}\~ao}
        % O preambulo deve conter o tipo do trabalho, o objetivo, 
				% o nome da institui��o, a �rea de concentra��o e op��o quando houver				
				\preambulo{Disserta\c{c}\~ao/Tese apresentada ao Instituto de Ci\^encias Matem\'aticas e de Computa\c{c}\~ao, Universidade de S\~ao Paulo - ICMC/USP, como parte dos requisitos para obten\c{c}\~ao do t\'itulo de Mestre/Doutor em Ci\^encias - Programa.}
				\notaficha{Disserta\c{c}\~ao/Tese (Mestrado/Doutorado - Programa)}
        }}				







