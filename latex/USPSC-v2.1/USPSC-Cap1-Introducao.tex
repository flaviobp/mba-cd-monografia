%% USPSC-Introducao.tex

% ----------------------------------------------------------
% Introdução (exemplo de capítulo sem numeração, mas presente no Sumário)
% ----------------------------------------------------------
\chapter[Introdução]{Introdução}

A inflação tem um impacto significativo na economia, afetando as decisões tomadas pelos agentes econômicos e na definição da taxa de juros pela autoridade monetária. Desde 1999, o Brasil adotou o regime de metas de inflação que busca estabelecer limites para a inflação medida em um ano com um valor central e um intervalo de tolerância definidos. Nesse contexto, a autoridade monetária trabalha para prever o comportamento da inflação futura e ajustar a política monetária para atingir a meta estabelecida e amparar as expectativas dos agentes econômicos em um sentido de convergência para o controle inflacionário. Isso é fundamental para a estabilidade econômica, influenciando decisões de investimento, o nível de emprego e o poder de compra das famílias.

As metas para a inflação são determinadas pelo Conselho Monetário Nacional (CMN) e tem como alvo o Índice de Preços ao Consumidor Amplo (IPCA), índice de preços amplamente conhecido e calculado mensalmente pelo Instituto Brasileiro de Geografia e Estatística (IBGE) que leva em consideração o custo de vida de famílias com renda mensal entre um e quarenta salários mínimos residentes nas regiões metropolitanas de Belém, Fortaleza, Recife, Salvador, Belo Horizonte, Rio de Janeiro, São Paulo, Curitiba, Vitória e Porto Alegre, Brasília e dos municípios de Goiânia, Campo Grande, Rio Branco, São Luís e Aracaju. 

Após a determinação e divulgação das metas de inflação e sua tolerância pelo CMN, cabe ao Banco Central (BACEN) o cumprimento da meta através de modificações em regulamentos e normas, com adoção de uma política monetária consoante com a meta a ser atingida que utiliza como instrumento a taxa de juros Selic.

Assim, a previsão da inflação futura é crucial, dada a incerteza do cenário econômico, e o Banco Central (BACEN) disponibiliza informações sobre projeções e  perspectivas em seus Relatórios de Inflação, bem como sintetiza pesquisas de expectativas de mercado publicadas no Boletim Focus. Nesse contexto, a utilização de métodos de aprendizagem de máquina é uma ferramenta valiosa para formuladores de políticas econômicas e agentes econômicos nas tomadas de decisões.

\section{Revisão bibliográfica}\label{sec-revisao-bibliografica}

Nos últimos anos, a utilização de técnicas de aprendizagem de máquina tem ganhado destaque em análises econômicas. \citeonline{varian2014big} destacou algumas limitações das técnicas convencionais de estatística e econometria, como regressões, aplicadas em análises econômicas para um grande conjunto de dados. O poder de ferramentas para um grande volume de dados, o extenso número de potenciais preditores e modelos com relações não lineares, foram as limitações apontadas pelo autor. Assim, técnicas de aprendizagem de máquinas, como redes neurais e árvores de decisão, podem ser mais eficazes em problemas com grande conjunto de dados e relações mais complexas.

Em questões de análise econômicas, as técnicas de aprendizagem de máquina vem ganhando espaço em tarefas de predição com grandes conjunto de dados, enquanto técnicas econométricas tradicionais estão ligadas a tarefas de inferência casual. Neste sentido, \citeonline{chakraborty2017machine} afirma métodos de aprendizagem de máquina e econometria podem ser vistos como uma extensão mutua um do outro, onde um problema de política economia pode ser divido em uma componente de previsão e outra de inferência casual.

No âmbito de previsão na macroeconomia, \citeonline{goulet2022machine} relaciona as características que fazem com que as técnicas de aprendizagem de máquina supere as abordagens convencionais macroeconométricas, como a não linearidade, regularização, validação cruzada e função de perda. O autor cita que a não linearidade é a principal característica das técnicas de aprendizagem de máquina, especialmente em cenários com altas incertezas, como estouros de bolhas ou estresse financeiro, para previsões em macroeconomia.

É notório o crescimento do uso de técnicas de aprendizagem de máquina em economia. \citeonline{athey2018impact} cita os impactos desta transformação, como novas áreas de pesquisa e trabalhos aplicados que utilizam aprendizagem de máquina. Essa mudança tem levado à transformação na formação de economistas, com a inclusão de cursos de ciência de dados e técnicas de aprendizagem de máquina nos currículos.

Podemos encontrar recentes trabalhos aplicados à economia, como em \citeonline{costa2021machine} que explora uma abordagem para previsão de preços de petróleo em curto e médio prazo utilizando métodos de árvores de regressão e procedimentos de regularização, como lasso e elastic net. \citeonline{hall2018machine} apresenta uma modelagem para previsão da taxa de desemprego utilizando o método de aprendizagem de máquina elastic net, onde se obtém melhores resultados no curto prazo do que em modelos tradicionais com modelos autorregressivos.

Considerando aplicações de técnicas de aprendizagem a macroeconomia, encontramos em \citeonline{chakraborty2017machine} exemplos de estudos de caso que destacam os métodos de aprendizagem de máquina em contextos de bancos centrais. Já para tarefas de predição, \citeonline{richardson2021nowcasting} faz um estudo de modelos aprendizagem de máquinas aplicados no contexto para previsão de crescimento real do PIB da Nova Zelândia, neste estudo foram utilizadas 600 variáveis, onde o resultado superou em alguns casos as abordagens tradicionais.   

No estudo realizado por \citeonline{dopke2017predicting}, é abordada a previsão de recessões na Alemanha através do uso de árvores de regressão. O autor destaca a complementaridade dessa análise com a abordagem probit, uma abordagem mais tradicional, para a análise prática dos ciclos de negócios. Além disso, é ressaltada a capacidade de identificar os efeitos marginais não lineares dos indicadores na probabilidade de recessão.

No contexto de previsão da inflação, \citeonline{medeiros2021forecasting} demonstra a aplicação do modelo de florestas aleatórias, obtendo resultados mais precisos na previsão da inflação nos EUA em comparação aos métodos tradicionais. O estudo organizou um ambiente rico em dados, utilizando 122 variáveis macroeconômicas mensais no período de janeiro de 1960 a dezembro de 2015. De acordo com o autor, o bom desempenho dos métodos de aprendizado de máquina se deve às não-linearidades e à possibilidade de seleção de variáveis.

No âmbito da previsão da inflação no Brasil, utilizando métodos de aprendizagem de máquina, temos o trabalho realizado por \citeonline{garcia2017real}, que empregou o método LASSO e 93 variáveis no período de janeiro de 2003 a dezembro de 2015. Esse estudo obteve um bom desempenho em horizontes de previsão mais curtos. Em outra pesquisa nessa área, desenvolvida por \citeonline{araujo2023machine}, foi realizado um comparativo entre métodos de aprendizagem de máquina e modelos econométricos tradicionais. Nesse estudo, mais de 501 séries macroeconômicas foram utilizadas para prever a inflação no Brasil. Os métodos baseados em árvore apresentaram bons resultados de desempenho.

\section{Objetivos do trabalho}\label{sec-objetivos}

Portanto, o objetivo deste trabalho, é aplicar métodos de aprendizagem de máquina na previsão da inflação no Brasil utilizando séries macroeconômicas divulgadas por organizações e institutos de pesquisas, como Banco Central e IBGE. Para isso, montaremos uma base de dados de séries temporais para prever a inflação a partir de APIs públicas, e iremos estudar a aplicação de métodos de aprendizagem de máquina em séries temporais em um contexto macroeconômico com finalidade de se obter modelos satisfatórios para predição da inflação quando comparados com os métodos tradicionais divulgados pelo Banco Central, como o Boletim Focus.

\section{Estrutura do trabalho}\label{sec-estrutura}

O presente trabalho foi estruturado em quatro capítulos. No primeiro capítulo, é realizada uma introdução com ênfase na importância e relevância do tema, além de fornecer uma revisão bibliográfica abrangente e objetivos de pesquisa delimitados. No segundo capítulo, é apresentada a metodologia aplicada no desenvolvimento deste trabalho, descrevendo detalhadamente a coleta de dados e os métodos teóricos empregados. Os resultados obtidos são discutidos no terceiro capítulo, que engloba uma análise descritiva dos dados coletados, bem como a utilização de métricas e comparações de desempenho entre diferentes métodos de aprendizagem de máquina aplicados à previsão da inflação. Por fim, no quarto capítulo, apresentamos as conclusões deste trabalho, resumindo os principais resultados obtidos e sugestões e indicações para trabalhos futuros, com novas possibilidades de estudo sobre o tema.