%% USPSC-pre-textual-OUTRO.tex
%% Camandos para defini��o do tipo de documento (tese ou disserta��o), �rea de concentra��o, op��o, pre�mbulo, titula��o 
%% referentes ao Programa de P�s-Gradua��o o IQSC
\instituicao{Nome da Unidade USP, Universidade de S\~ao Paulo}
\unidade{NOME DA UNIDADE USP}
\unidademin{Nome da Unidade USP}
\universidademin{Universidade de S\~ao Paulo}

\notafolharosto{Vers\~ao original}
%Para vers�o original em ingl�s, comente do comando/declara��o 
%     acima(inclua % antes do comando acima) e tire a % do 
%     comando/declara��o abaixo no idioma do texto
%\notafolharosto{Original version} 
%Para vers�o corrigida, comente do comando/declara��o da 
%     vers�o original acima (inclua % antes do comando acima) 
%     e tire a % do comando/declara��o de um dos comandos 
%     abaixo em conformidade com o idioma do texto
%\notafolharosto{Vers\~ao corrigida \\(Vers\~ao original dispon\'ivel na Unidade que aloja o Programa)}
%\notafolharosto{Corrected version \\(Original version available on the Program Unit)}

% ---
% dados complementares para CAPA e FOLHA DE ROSTO
% ---
\universidade{UNIVERSIDADE DE S\~AO PAULO}
\titulo{Modelo para teses e disserta\c{c}\~oes em \LaTeX\ utilizando a classe USPSC}
\titleabstract{Model for theses and dissertations in \LaTeX\ using the USPSC class}
\autor{Jos\'e da Silva}
\autorficha{Silva, Jos\'e da}
\autorabr{SILVA, J.}

\cutter{S856m}
% Para gerar a ficha catalogr�fica sem o C�digo Cutter, basta 
% incluir uma % na linha acima e tirar a % da linha abaixo
%\cutter{ }

\local{S\~ao Carlos}
\data{2016}
% Quando for Orientador, basta incluir uma % antes do comando abaixo
\renewcommand{\orientadorname}{Orientadora:}
% Quando for Coorientadora, basta tirar a % utilizar o comando abaixo
%\newcommand{\coorientadorname}{Coorientador:}
\orientador{Profa. Dra. Elisa Gon\c{c}alves Rodrigues}
\orientadorcorpoficha{orientadora Elisa Gon\c{c}alves Rodrigues}
\orientadorficha{Rodrigues, Elisa Gon\c{c}alves, orient}
%Se houver co-orientador, inclua % antes das duas linhas (antes dos comandos \orientadorcorpoficha e \orientadorficha) 
%          e tire a % antes dos 3 comandos abaixo
%\coorientador{Prof. Dr. Jo\~ao Alves Serqueira}
%\orientadorcorpoficha{orientadora Elisa Gon\c{c}alves Rodrigues ;  co-orientador Jo\~ao Alves Serqueira}
%\orientadorficha{Rodrigues, Elisa Gon\c{c}alves, orient. II. Serqueira, Jo\~ao Alves, co-orient}

\notaautorizacao{AUTORIZO A REPRODU\c{C}\~AO E DIVULGA\c{C}\~AO TOTAL OU PARCIAL DESTE TRABALHO, POR QUALQUER MEIO CONVENCIONAL OU ELETR\^ONICO PARA FINS DE ESTUDO E PESQUISA, DESDE QUE CITADA A FONTE.}
\notabib{Ficha catalogr\'afica elaborada pela Biblioteca da Unidade USP, com os dados fornecidos pelo(a) autor(a)}

\newcommand{\programa}[1]{

% DOUTRO ==========================================================================
    \ifthenelse{\equal{#1}{DOUTRO}}{
     	  \area{Nome da \'Area}
				\tipotrabalho{Tese (Doutorado)}
				%\opcao{Nome da Op��o}
        % O preambulo deve conter o tipo do trabalho, o objetivo, 
				% o nome da institui��o e a �rea de concentra��o 
				\preambulo{Tese apresentada ao Programa de P\'os-Gradua\c{c}\~ao em XXXXXXX da Unidade USP, Universidade de S\~ao Paulo, como parte dos requisitos para a obten\c{c}\~ao do t\'itulo de Doutor em YYYYYYYYYYY.}
				\notaficha{Tese (Doutorado - Programa de P\'os-Gradua\c{c}\~ao em XXXXXXX e \'Area de Concentra\c{c}\~ao em ~\imprimirarea)}
    }{
% MOUTRO ===========================================================================
        \ifthenelse{\equal{#1}{MOUTRO}}{
        \area{Nome da \'Area}
				\tipotrabalho{Disserta\c{c}\~ao (Mestrado)}
				%\opcao{Nome da Op��o}
     		% O preambulo deve conter o tipo do trabalho, o objetivo, 
				% o nome da institui��o e a �rea de concentra��o 
				\preambulo{Disserta\c{c}\~ao apresentada ao Programa de P\'os-Gradua\c{c}\~ao em XXXXXXX da Unidade USP, Universidade de S\~ao Paulo, como parte dos requisitos para a obten\c{c}\~ao do t\'itulo de Mestre em YYYYYYYYYYY.}
				\notaficha{Disserta\c{c}\~ao (Mestrado - Programa de P\'os-Gradua\c{c}\~ao em XXXXXXX e \'Area de Concentra\c{c}\~ao em ~\imprimirarea)}
        }{
% Outros
		\tipotrabalho{Disserta\c{c}\~ao/Tese (Mestrado/Doutorado)}
		\area{Nome da \'Area}
		\opcao{Nome da Op\c{c}\~ao}
				% O preambulo deve conter o tipo do trabalho, o objetivo, 
				% o nome da institui��o e a �rea de concentra��o 
				\preambulo{Disserta\c{c}\~ao/Tese apresentada ao Programa de P\'os-Gradua\c{c}\~ao em XXXXXXX da Unidade USP, Universidade de S\~ao Paulo, como parte dos requisitos para a obten\c{c}\~ao do t\'itulo de Mestre/Doutor em YYYYYYYYYYY.}
				\notaficha{Disserta\c{c}\~ao/Tese (Mestrado/Doutorado - Programa de P\'os-Gradua\c{c}\~ao em XXXXXXX e \'Area de Concentra\c{c}\~ao em ~\imprimirarea)}
  
        }}}
				